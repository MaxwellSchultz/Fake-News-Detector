% This must be in the first 5 lines to tell arXiv to use pdfLaTeX, which is strongly recommended.
\pdfoutput=1
% In particular, the hyperref package requires pdfLaTeX in order to break URLs across lines.

\documentclass[11pt]{article}

% Remove the "review" option to generate the final version.
 \usepackage{ACL2023}

% Standard package includes
\usepackage{times}
\usepackage{latexsym}

% For proper rendering and hyphenation of words containing Latin characters (including in bib files)
\usepackage[T1]{fontenc}
% For Vietnamese characters
% \usepackage[T5]{fontenc}
% See https://www.latex-project.org/help/documentation/encguide.pdf for other character sets

% This assumes your files are encoded as UTF8
\usepackage[utf8]{inputenc}

% This is not strictly necessary, and may be commented out.
% However, it will improve the layout of the manuscript,
% and will typically save some space.
\usepackage{microtype}

% This is also not strictly necessary, and may be commented out.
% However, it will improve the aesthetics of text in
% the typewriter font.
\usepackage{inconsolata}


% If the title and author information does not fit in the area allocated, uncomment the following
%
%\setlength\titlebox{<dim>}
%
% and set <dim> to something 5cm or larger.

\title{Fake News Detection Using Neural Network Language Models}

% Author information can be set in various styles:
% For several authors from the same institution:
% \author{Author 1 \and ... \and Author n \\
%         Address line \\ ... \\ Address line}
% if the names do not fit well on one line use
%         Author 1 \\ {\bf Author 2} \\ ... \\ {\bf Author n} \\
% For authors from different institutions:
% \author{Author 1 \\ Address line \\  ... \\ Address line
%         \And  ... \And
%         Author n \\ Address line \\ ... \\ Address line}
% To start a seperate ``row'' of authors use \AND, as in
% \author{Author 1 \\ Address line \\  ... \\ Address line
%         \AND
%         Author 2 \\ Address line \\ ... \\ Address line \And
%         Author 3 \\ Address line \\ ... \\ Address line}

\author{Christopher Farrer \\
  \texttt{farrerc@wwu.edu} \\\And
  Maxwell Schultz \\
  \texttt{schultm8@wwu.edu} \\\And
  Alex Sitzman \\
  \texttt{sitzmaa@wwu.edu} \\}

\begin{document}
\maketitle
\begin{abstract}
This project aims to classify news article titles as sarcastic or non-sarcastic based on their content and context. Initially focused on sarcasm detection, the project pivoted to fake news detection due to dataset limitations. We used datasets from Kaggle and implemented multiple models, including Neural Bag of Words, Recurrent Neural Networks (RNNs), and Transformers. Our findings reveal that while Transformer models exhibit high accuracy, they struggle with nuanced contexts such as satire, and result in longer training times over other models. This report details our methodology, experimental results, and future directions for improving sarcasm detection.
\end{abstract}

\section{Introduction}

Neural Networks have become pivotal in various NLP tasks due to their ability to learn complex patterns from large datasets. This project aimed to leverage neural networks and traditional machine learning models to detect sarcasm in news headlines. Sarcasm detection holds significant ethical benefits, as it can improve the understanding and moderation of online communication, reducing misunderstandings and promoting healthier online interactions. However, limitations in the dataset necessitated a pivot to fake news detection. Fake news detection carries ethical benefits as well, as it helps combat misinformation, which can influence public opinion and decision-making processes. By ensuring the authenticity of news articles, such systems contribute to a more informed and rational public discourse. This paper discusses our approach, experimental setup, results, and the implications of our findings.

\section{Related Work}


Fake News Detection
The proliferation of fake news has become a major concern, impacting political, economic, and social landscapes. \cite{Agarwal2019} conducted a comprehensive study on fake news detection using a combination of NLP techniques and machine learning classifiers. They employed methods such as bag-of-words, n-grams, count vectorizer, and TF-IDF, training their dataset on five different classifiers. Their findings indicated that precision, recall, and F1 scores are critical metrics in determining the most effective model for fake news detection .

Sarcasm Detection
Sarcasm detection poses unique challenges due to its inherently ambiguous nature. \cite{Jain2017} explored sarcasm detection in tweets by utilizing ensemble-based approaches, specifically a voted ensemble classifier and a random forest classifier. Their research highlighted the difficulty in detecting sarcasm due to the evolving nature of language and the use of emoticons, which can alter the polarity of text. Their methodology diverged from traditional sentiment analysis by focusing on the presence of positive sentiment attached to negative situations, thus improving the detection accuracy in social media contexts .

Similarly, \cite{Baroiu2022} provided a systematic literature review on automatic sarcasm detection, tracing its evolution from 2010 to the present. They emphasized the growing popularity of multi-modal approaches and transformer-based architectures in recent years. Their work not only critiqued past research but also proposed future directions, underscoring the necessity for advanced models capable of handling the complexities of sarcasm in natural language .



\section{Approach}


Our approach utilizes the PyTorch library to implement and train various sequence classification models for sentiment analysis, specifically targeting movie reviews. The implementation details and models are based on the tutorials provided by \cite{Trevett2023}, which offer a comprehensive guide on using PyTorch for sentiment analysis tasks. Below, we outline the main steps and models used in our approach.
\newline
Models and Tutorials
\\

Neural Bag of Words:

This tutorial introduces the basic workflow of a sequence classification project using a neural bag-of-words model. It covers data loading and preprocessing using the datasets and torchtext libraries. The model is simple yet effective for understanding the foundational concepts of sequence classification.

Recurrent Neural Networks (RNN):

Building on the basic workflow, this tutorial focuses on improving the model's performance by switching to a recurrent neural network (RNN) model. Specifically, it implements a long short-term memory (LSTM) RNN, which is one of the most commonly used variants of RNNs due to its ability to handle long-range dependencies in sequential data.

Transformers:

The final tutorial demonstrates how to use the transformers library to load a pre-trained transformer model, specifically BERT (Bidirectional Encoder Representations from Transformers). BERT, introduced in the paper by Devlin et al., provides high performance for various NLP tasks, including sequence classification. The tutorial covers loading the pre-trained BERT model and fine-tuning it for sentiment analysis.

Datasets

\cite{sarcasm-dataset}
\cite{fake-news-dataset}


\section{Document Body}

\subsection{Footnotes}

Footnotes are inserted with the \verb|\footnote| command.\footnote{This is a footnote.}

\subsection{Tables and figures}

See Table~\ref{tab:accents} for an example of a table and its caption.
\textbf{Do not override the default caption sizes.}

\subsection{Hyperlinks}

Users of older versions of \LaTeX{} may encounter the following error during compilation: 
\begin{quote}
\tt\verb|\pdfendlink| ended up in different nesting level than \verb|\pdfstartlink|.
\end{quote}
This happens when pdf\LaTeX{} is used and a citation splits across a page boundary. The best way to fix this is to upgrade \LaTeX{} to 2018-12-01 or later.

\subsection{References}



The \LaTeX{} and Bib\TeX{} style files provided roughly follow the American Psychological Association format.
If your own bib file is named \texttt{custom.bib}, then placing the following before any appendices in your \LaTeX{} file will generate the references section for you:
\begin{quote}
\begin{verbatim}
\bibliographystyle{acl_natbib}
\bibliography{custom}
\end{verbatim}
\end{quote}
You can obtain the complete ACL Anthology as a Bib\TeX{} file from \url{https://aclweb.org/anthology/anthology.bib.gz}.
To include both the Anthology and your own .bib file, use the following instead of the above.
\begin{quote}
\begin{verbatim}
\bibliographystyle{acl_natbib}
\bibliography{anthology,custom}
\end{verbatim}
\end{quote}
Please see Section~\ref{sec:bibtex} for information on preparing Bib\TeX{} files.

\subsection{Appendices}

Use \verb|\appendix| before any appendix section to switch the section numbering over to letters. See Appendix~\ref{sec:appendix} for an example.

\section{Bib\TeX{} Files}
\label{sec:bibtex}

Unicode cannot be used in Bib\TeX{} entries, and some ways of typing special characters can disrupt Bib\TeX's alphabetization. The recommended way of typing special characters is shown in Table~\ref{tab:accents}.

Please ensure that Bib\TeX{} records contain DOIs or URLs when possible, and for all the ACL materials that you reference.
Use the \verb|doi| field for DOIs and the \verb|url| field for URLs.
If a Bib\TeX{} entry has a URL or DOI field, the paper title in the references section will appear as a hyperlink to the paper, using the hyperref \LaTeX{} package.

\section*{Limitations}
ACL 2023 requires all submissions to have a section titled ``Limitations'', for discussing the limitations of the paper as a complement to the discussion of strengths in the main text. This section should occur after the conclusion, but before the references. It will not count towards the page limit.
The discussion of limitations is mandatory. Papers without a limitation section will be desk-rejected without review.

While we are open to different types of limitations, just mentioning that a set of results have been shown for English only probably does not reflect what we expect. 
Mentioning that the method works mostly for languages with limited morphology, like English, is a much better alternative.
In addition, limitations such as low scalability to long text, the requirement of large GPU resources, or other things that inspire crucial further investigation are welcome.

\section*{Ethics Statement}
Scientific work published at ACL 2023 must comply with the ACL Ethics Policy.\footnote{\url{https://www.aclweb.org/portal/content/acl-code-ethics}} We encourage all authors to include an explicit ethics statement on the broader impact of the work, or other ethical considerations after the conclusion but before the references. The ethics statement will not count toward the page limit (8 pages for long, 4 pages for short papers).

\section*{Acknowledgements}
This document has been adapted by Jordan Boyd-Graber, Naoaki Okazaki, Anna Rogers from the style files used for earlier ACL, EMNLP and NAACL proceedings, including those for
EACL 2023 by Isabelle Augenstein and Andreas Vlachos,
EMNLP 2022 by Yue Zhang, Ryan Cotterell and Lea Frermann,
ACL 2020 by Steven Bethard, Ryan Cotterell and Rui Yan,
ACL 2019 by Douwe Kiela and Ivan Vuli\'{c},
NAACL 2019 by Stephanie Lukin and Alla Roskovskaya, 
ACL 2018 by Shay Cohen, Kevin Gimpel, and Wei Lu, 
NAACL 2018 by Margaret Mitchell and Stephanie Lukin,
Bib\TeX{} suggestions for (NA)ACL 2017/2018 from Jason Eisner,
ACL 2017 by Dan Gildea and Min-Yen Kan, NAACL 2017 by Margaret Mitchell, 
ACL 2012 by Maggie Li and Michael White, 
ACL 2010 by Jing-Shin Chang and Philipp Koehn, 
ACL 2008 by Johanna D. Moore, Simone Teufel, James Allan, and Sadaoki Furui, 
ACL 2005 by Hwee Tou Ng and Kemal Oflazer, 
ACL 2002 by Eugene Charniak and Dekang Lin, 
and earlier ACL and EACL formats written by several people, including
John Chen, Henry S. Thompson and Donald Walker.
Additional elements were taken from the formatting instructions of the \emph{International Joint Conference on Artificial Intelligence} and the \emph{Conference on Computer Vision and Pattern Recognition}.

% Entries for the entire Anthology, followed by custom entries
\bibliography{anthology,custom}
\bibliographystyle{acl_natbib}

\appendix

\section{Example Appendix}
\label{sec:appendix}

This is a section in the appendix.

\end{document}
